%% bare_jrnl_seminar.tex
%%
%% This is a slightly modified version of the original IEEEtran example bare_jrnl.tex 
%% to meet the needs for the "advanced seminar for security in information technology"
%% at the institute for security in information technology, TUM. 
%% This template is also applicable for writing German texts.
%% 
%% April 2011, Hermann Seuschek
%%


%% bare_jrnl.tex
%% V1.3
%% 2007/01/11
%% by Michael Shell
%% see http://www.michaelshell.org/
%% for current contact information.
%%
%% This is a skeleton file demonstrating the use of IEEEtran.cls
%% (requires IEEEtran.cls version 1.7 or later) with an IEEE journal paper.
%%
%% Support sites:
%% http://www.michaelshell.org/tex/ieeetran/
%% http://www.ctan.org/tex-archive/macros/latex/contrib/IEEEtran/
%% and
%% http://www.ieee.org/


% *** Authors should verify (and, if needed, correct) their LaTeX system  ***
% *** with the testflow diagnostic prior to trusting their LaTeX platform ***
% *** with production work. IEEE's font choices can trigger bugs that do  ***
% *** not appear when using other class files.                            ***
% The testflow support page is at:
% http://www.michaelshell.org/tex/testflow/


%%*************************************************************************
%% Legal Notice:
%% This code is offered as-is without any warranty either expressed or
%% implied; without even the implied warranty of MERCHANTABILITY or
%% FITNESS FOR A PARTICULAR PURPOSE! 
%% User assumes all risk.
%% In no event shall IEEE or any contributor to this code be liable for
%% any damages or losses, including, but not limited to, incidental,
%% consequential, or any other damages, resulting from the use or misuse
%% of any information contained here.
%%
%% All comments are the opinions of their respective authors and are not
%% necessarily endorsed by the IEEE.
%%
%% This work is distributed under the LaTeX Project Public License (LPPL)
%% ( http://www.latex-project.org/ ) version 1.3, and may be freely used,
%% distributed and modified. A copy of the LPPL, version 1.3, is included
%% in the base LaTeX documentation of all distributions of LaTeX released
%% 2003/12/01 or later.
%% Retain all contribution notices and credits.
%% ** Modified files should be clearly indicated as such, including  **
%% ** renaming them and changing author support contact information. **
%%
%% File list of work: IEEEtran.cls, IEEEtran_HOWTO.pdf, bare_adv.tex,
%%                    bare_conf.tex, bare_jrnl.tex, bare_jrnl_compsoc.tex
%%*************************************************************************

% Note that the a4paper option is mainly intended so that authors in
% countries using A4 can easily print to A4 and see how their papers will
% look in print - the typesetting of the document will not typically be
% affected with changes in paper size (but the bottom and side margins will).
% Use the testflow package mentioned above to verify correct handling of
% both paper sizes by the user's LaTeX system.
%
% Also note that the "draftcls" or "draftclsnofoot", not "draft", option
% should be used if it is desired that the figures are to be displayed in
% draft mode.
%
\documentclass[10pt,        % Don't change the font size!
               a4paper,     % Don't change the paper size!
               journal,     % Journal paper format
%               draft       % Enable this parameter to get a draft version.
               ]{IEEEtran}
\makeatletter


\def\markboth#1#2{\def\leftmark{\@IEEEcompsoconly{\sffamily}\MakeUppercase{\protect#1}}%
\def\rightmark{\@IEEEcompsoconly{\sffamily}\MakeUppercase{\protect#2}}}
\makeatother

%
% Use this package for new German hyphenation and to set some 
% captions to German (e.g. Abstract -> Zusammenfassung)
\usepackage[ngerman]{babel}

%
% Select input file coding Latin1 or UTF8.
% See "http://en.wikipedia.org/wiki/Latin1" and "http://en.wikipedia.org/wiki/Utf8"
% for more information.
%
%\usepackage[latin1]{inputenc}
\usepackage[utf8]{inputenc}

\usepackage[T1]{fontenc}

%
% If IEEEtran.cls has not been installed into the LaTeX system files,
% manually specify the path to it like:
% \documentclass[journal]{../sty/IEEEtran}

% Some very useful LaTeX packages include:
% (uncomment the ones you want to load)


% *** MISC UTILITY PACKAGES ***
%
%\usepackage{ifpdf}
% Heiko Oberdiek's ifpdf.sty is very useful if you need conditional
% compilation based on whether the output is pdf or dvi.
% usage:
% \ifpdf
%   % pdf code
% \else
%   % dvi code
% \fi
% The latest version of ifpdf.sty can be obtained from:
% http://www.ctan.org/tex-archive/macros/latex/contrib/oberdiek/
% Also, note that IEEEtran.cls V1.7 and later provides a builtin
% \ifCLASSINFOpdf conditional that works the same way.
% When switching from latex to pdflatex and vice-versa, the compiler may
% have to be run twice to clear warning/error messages.


\usepackage{cite}

\usepackage[pdftex]{graphicx}

\usepackage[cmex10]{amsmath}







%\usepackage[caption=false]{caption}
%\usepackage[font=footnotesize]{subfig}
% subfig.sty, also written by Steven Douglas Cochran, is the modern
% replacement for subfigure.sty. However, subfig.sty requires and
% automatically loads Axel Sommerfeldt's caption.sty which will override
% IEEEtran.cls handling of captions and this will result in nonIEEE style
% figure/table captions. To prevent this problem, be sure and preload
% caption.sty with its "caption=false" package option. This is will preserve
% IEEEtran.cls handing of captions. Version 1.3 (2005/06/28) and later 
% (recommended due to many improvements over 1.2) of subfig.sty supports
% the caption=false option directly:
%\usepackage[caption=false,font=footnotesize]{subfig}
%
% The latest version and documentation can be obtained at:
% http://www.ctan.org/tex-archive/macros/latex/contrib/subfig/
% The latest version and documentation of caption.sty can be obtained at:
% http://www.ctan.org/tex-archive/macros/latex/contrib/caption/




% *** FLOAT PACKAGES ***
%
%\usepackage{fixltx2e}
% fixltx2e, the successor to the earlier fix2col.sty, was written by
% Frank Mittelbach and David Carlisle. This package corrects a few problems
% in the LaTeX2e kernel, the most notable of which is that in current
% LaTeX2e releases, the ordering of single and double column floats is not
% guaranteed to be preserved. Thus, an unpatched LaTeX2e can allow a
% single column figure to be placed prior to an earlier double column
% figure. The latest version and documentation can be found at:
% http://www.ctan.org/tex-archive/macros/latex/base/



%\usepackage{stfloats}
% stfloats.sty was written by Sigitas Tolusis. This package gives LaTeX2e
% the ability to do double column floats at the bottom of the page as well
% as the top. (e.g., "\begin{figure*}[!b]" is not normally possible in
% LaTeX2e). It also provides a command:
%\fnbelowfloat
% to enable the placement of footnotes below bottom floats (the standard
% LaTeX2e kernel puts them above bottom floats). This is an invasive package
% which rewrites many portions of the LaTeX2e float routines. It may not work
% with other packages that modify the LaTeX2e float routines. The latest
% version and documentation can be obtained at:
% http://www.ctan.org/tex-archive/macros/latex/contrib/sttools/
% Documentation is contained in the stfloats.sty comments as well as in the
% presfull.pdf file. Do not use the stfloats baselinefloat ability as IEEE
% does not allow \baselineskip to stretch. Authors submitting work to the
% IEEE should note that IEEE rarely uses double column equations and
% that authors should try to avoid such use. Do not be tempted to use the
% cuted.sty or midfloat.sty packages (also by Sigitas Tolusis) as IEEE does
% not format its papers in such ways.


%\ifCLASSOPTIONcaptionsoff
%  \usepackage[nomarkers]{endfloat}
% \let\MYoriglatexcaption\caption
% \renewcommand{\caption}[2][\relax]{\MYoriglatexcaption[#2]{#2}}
%\fi
% endfloat.sty was written by James Darrell McCauley and Jeff Goldberg.
% This package may be useful when used in conjunction with IEEEtran.cls'
% captionsoff option. Some IEEE journals/societies require that submissions
% have lists of figures/tables at the end of the paper and that
% figures/tables without any captions are placed on a page by themselves at
% the end of the document. If needed, the draftcls IEEEtran class option or
% \CLASSINPUTbaselinestretch interface can be used to increase the line
% spacing as well. Be sure and use the nomarkers option of endfloat to
% prevent endfloat from "marking" where the figures would have been placed
% in the text. The two hack lines of code above are a slight modification of
% that suggested by in the endfloat docs (section 8.3.1) to ensure that
% the full captions always appear in the list of figures/tables - even if
% the user used the short optional argument of \caption[]{}.
% IEEE papers do not typically make use of \caption[]'s optional argument,
% so this should not be an issue. A similar trick can be used to disable
% captions of packages such as subfig.sty that lack options to turn off
% the subcaptions:
% For subfig.sty:
% \let\MYorigsubfloat\subfloat
% \renewcommand{\subfloat}[2][\relax]{\MYorigsubfloat[]{#2}}
% For subfigure.sty:
% \let\MYorigsubfigure\subfigure
% \renewcommand{\subfigure}[2][\relax]{\MYorigsubfigure[]{#2}}
% However, the above trick will not work if both optional arguments of
% the \subfloat/subfig command are used. Furthermore, there needs to be a
% description of each subfigure *somewhere* and endfloat does not add
% subfigure captions to its list of figures. Thus, the best approach is to
% avoid the use of subfigure captions (many IEEE journals avoid them anyway)
% and instead reference/explain all the subfigures within the main caption.
% The latest version of endfloat.sty and its documentation can obtained at:
% http://www.ctan.org/tex-archive/macros/latex/contrib/endfloat/
%
% The IEEEtran \ifCLASSOPTIONcaptionsoff conditional can also be used
% later in the document, say, to conditionally put the References on a 
% page by themselves.





% *** PDF, URL AND HYPERLINK PACKAGES ***
%
%\usepackage{url}
% url.sty was written by Donald Arseneau. It provides better support for
% handling and breaking URLs. url.sty is already installed on most LaTeX
% systems. The latest version can be obtained at:
% http://www.ctan.org/tex-archive/macros/latex/contrib/misc/
% Read the url.sty source comments for usage information. Basically,
% \url{my_url_here}.





% *** Do not adjust lengths that control margins, column widths, etc. ***
% *** Do not use packages that alter fonts (such as pslatex).         ***
% There should be no need to do such things with IEEEtran.cls V1.6 and later.
% (Unless specifically asked to do so by the journal or conference you plan
% to submit to, of course. )


% correct bad hyphenation here
\hyphenation{op-tical net-works semi-conduc-tor}


\begin{document}
%
% paper title
% can use linebreaks \\ within to get better formatting as desired
\title{XTR Public Key System}
%
%
% author names and IEEE memberships
% note positions of commas and nonbreaking spaces ( ~ ) LaTeX will not break
% a structure at a ~ so this keeps an author's name from being broken across
% two lines.
% use \thanks{} to gain access to the first footnote area
% a separate \thanks must be used for each paragraph as LaTeX2e's \thanks
% was not built to handle multiple paragraphs
%

\author{Sebastian~Schleemilch}% <-this % stops a space
%\thanks{M. Shell is with the Department
%of Electrical and Computer Engineering, Georgia Institute of Technology, Atlanta,
%GA, 30332 USA e-mail: (see http://www.michaelshell.org/contact.html).}% <-this % stops a space
%\thanks{J. Doe and J. Doe are with Anonymous University.}% <-this % stops a space
%\thanks{Manuscript received April 19, 2005; revised January 11, 2007.}}

% note the % following the last \IEEEmembership and also \thanks - 
% these prevent an unwanted space from occurring between the last author name
% and the end of the author line. i.e., if you had this:
% 
% \author{....lastname \thanks{...} \thanks{...} }
%                     ^------------^------------^----Do not want these spaces!
%
% a space would be appended to the last name and could cause every name on that
% line to be shifted left slightly. This is one of those "LaTeX things". For
% instance, "\textbf{A} \textbf{B}" will typeset as "A B" not "AB". To get
% "AB" then you have to do: "\textbf{A}\textbf{B}"
% \thanks is no different in this regard, so shield the last } of each \thanks
% that ends a line with a % and do not let a space in before the next \thanks.
% Spaces after \IEEEmembership other than the last one are OK (and needed) as
% you are supposed to have spaces between the names. For what it is worth,
% this is a minor point as most people would not even notice if the said evil
% space somehow managed to creep in.



% The paper headers
\markboth{Hauptseminar Sicherheit in der Informationstechnik, Sommersemester 2015}%
%\markboth{Advanced Seminar for Security in Information Technology, Summer Term 2011}%
{Sebastian Schleemilch: XTR Public Key System}

%\markboth{Journal of \LaTeX\ Class Files,~Vol.~6, No.~1, January~2007}%
%{Shell \MakeLowercase{\textit{et al.}}: Bare Demo of IEEEtran.cls for Journals}
% The only time the second header will appear is for the odd numbered pages
% after the title page when using the twoside option.
% 
% *** Note that you probably will NOT want to include the author's ***
% *** name in the headers of peer review papers.                   ***
% You can use \ifCLASSOPTIONpeerreview for conditional compilation here if
% you desire.




% If you want to put a publisher's ID mark on the page you can do it like
% this:
%\IEEEpubid{0000--0000/00\$00.00~\copyright~2007 IEEE}
% Remember, if you use this you must call \IEEEpubidadjcol in the second
% column for its text to clear the IEEEpubid mark.



% use for special paper notices
%\IEEEspecialpapernotice{(Invited Paper)}




% make the title area
\maketitle


\begin{abstract}
%\boldmath
Viele der heutigen Kryptosystemen basieren auf der Unlösbarkeit des diskreten Logarithmus Problems. 
Zu dieser Kategorie gehört DSA für die Erzeugung von digitalen Signaturen, der Diffie-Hellman Schlüsseltausch sowie das ElGamal-Signaturverfahren.
Das hier vorgestellte XTR-Verfahren kann auf alle gängigen Verfahren angewendet werden die auf dem diskreten Logarithmus Problem über
einer multiplikativen Gruppe eines endlichen Körpers basieren. Das Ziel von XTR ist eine kompaktere Darstellung des endlichen Körpers mit Hilfe einer Untergruppe.
Um die Alltagstauglichkeit von Kryptosystemen zu erhöhen, sollten sie so unauffällig wie möglich in die Anwendung integriert werden können. XTR leistet hier einen guten Beitrag.
Mit XTR sind die zu übertragenen Schlüssellängen (=Datenmenge), bei nachweislich gleichbleibender Sicherheit, kürzer. Außerdem sind die erforderlichen Berechnungen
für die Kryptosysteme effizienter als bei vergleichbaren Verfahren. Somit bietet sich das Verfahren auch für die Verwendung in Low-Power-Anwendungen wie z.B. SmartCards an.
\end{abstract}


\section{Einleitung}
\label{sec:einleitung}
Der Wunsch nach der Verschlüsselung von digitalen Daten wird sowohl im privaten als auch im geschäftlichen Bereich immer größer.
Ausschlaggebend für die Verwendung von Verschlüsselungstechnologien sind deren Sicherheit sowie deren Benutzbarkeit.
Grundsätzlich gibt es zwei Verfahren um eine Ende-zu-Ende Verschlüsselung umzusetzen.
Es gibt die symmetrische Verschlüsselung, bei der sich Sender \emph{A} und Empfänger \emph{B} auf einen gemeinsamen privaten Schlüssel einigen. Diesen Schlüssel verwenden die Teilnehmer um geheime Nachrichten auszutauschen. Das Problem liegt hier bei dem sicheren Austausch dieses Schlüssels, ohne dass ein Angreifer die Möglichkeit hat den Schlüssel mitzulesen. Der orthogonale Ansatz ist die Verwendung von Schlüsselpaaren pro Gesprächsteilnehmer. Sowohl Teilnehmer \emph{A} als auch \emph{B} erzeugen jeweils einen öffentlichen sowie einen privaten Schlüssel. Der öffentliche Teil des Schlüsselpaares kann in einer vertrauenswürdigen, für jeden Teilnehmer zugänglichen, Datenbank gespeichert werden um "`Man-in-the-middle"''-Angriffe zu vermeiden. Möchte \emph{A} eine Nachricht an \emph{B} schicken, besorgt sich \emph{A} den öffentlichen Schlüssel von \emph{B} und verschlüsselt mit diesem die Nachricht \emph{m}. Nur \emph{B} kann diese Nachricht mit dem nur \emph{B} bekannten privaten Schlüssel lesen.

\section{Mathematische Grundlagen der Gruppen und der Diffie-Hellman Schlüsselaustausch}
\label{sec:dh}
Diffie-Hellman ist ein Schlüsselaustauschverfahren (DH) mit dessen Hilfe Teilnehmer \emph{A} und \emph{B} ihren geheimen Schlüssel für eine zukünftig symmetrische Verschlüsselung über einen unsicheren Übertragungskanal austauschen können. Obwohl ein Angreifer möglicherweise die gesamte Kommunikation mithört, ist er nicht in der Lage sich den geheimen Schlüssel zu erzeugen. Dieses Verfahren und die dabei verwendeten Elemente sind eine
gute Grundlage für das Verständnis von XTR, da sich XTR auch für dieses Verfahren eignet.

Für den DH-Schlüsselaustausch benötigt man eine Primzahl \emph{p}, sowie eine \emph{multiplikative zyklische Gruppe $Z_p^*$} über dieser Primzahl.

Eine Gruppe im mathematischen Sinn besteht allgemein aus Elementen die sich Verknüpfen lassen, wie z.B. durch die Addition "`\emph{+}"', 
Multiplikation "`\emph{$\cdot$}"' oder der Division "`\emph{\%}"' aber auch durch komplexere Verknüpfungen wie der Berechnung "`$(a+b) \bmod c$"'. Die Gruppe muss außerdem folgende Eigenschaften besitzen\cite{gruppe}:

\begin{itemize}
	\item Assoziativität bezüglich der Verknüpfung $\circ$: $a \circ (b \circ c) = (a \circ b) \circ c$
	\item Es muss für jedes Element ein neutrales Element \emph{b} geben so dass mit der Verknüpfung $\circ$ gilt: $a \circ b = a$ 
	\item Es muss für jedes Element ein inverses Element \emph{b} geben so dass mit der Verknüpfung $\circ$ gilt: $a \circ b = \emph{neutrales Element}$
\end{itemize}

So besteht beispielsweise $Z_{10}$ aus den Elementen $\{0,1,2,3,4,5,6,7,8,9\}$. Diese Elemente bilden bezüglich der Addition mod 10 eine Gruppe, nicht aber bezüglich der Multiplikation mod 10. Hier gibt es für einige Elemente kein inverses Element um das neutrale Element der Multiplikation, also die Zahl 1 zu errechnen. 
Sei $x \in \{0,2,4,5,6,8\}$ und $y \in Z_{10}$. Für eine multiplikative Gruppe muss für alle Elemente die Eigenschaft $x * y = 1 \bmod 10$ gelten.
$x$ enthält in diesem Beispiel genau die Elemente ohne Lösung.
Entfernt man die Elemente $x$ aus $Z_{10}$, verbleibt dann also \emph{multiplikative Gruppe $Z_{10}^*$} $= \{1,3,7,9\}$\cite{mulgruppe}.

Allgemein besteht eine Gruppe $Z_n^*$ aus Elementen die zu \emph{n} Teilerfremd sind. Ist \emph{n} eine Primzahl \emph{p},
so besitzt diese Gruppe also $\emph{n}-1$ Elemente (z.B. $Z_7^* = \{1,2,3,4,5,6\}$).

Eine zyklische Gruppe liegt vor, wenn die Gruppe aus den Potenzen eines einzelnen Elementes erzeugt werden kann. Beispielsweise lässt sich die zyklische multiplikative Gruppe $Z_7^* = \{1,2,3,4,5,6\}$ durch Exponenten des Elements $5 \bmod 7$ erzeugen ($\{ 5^1, 5^2, 5^3, 5^4, 5^5, 5^6\}  = \{5, 4, 6, 2, 3, 1\} = \{1, 2, 3, 4, 5, 6\}$). Die Zahl \emph{5} aus diesem Beispiel wird Generator \emph{g} genannt. Eine Gruppe $Z_n^*$ ist außerdem auf jeden Fall zyklisch, falls $n = 2, 4, p^k, 2p^k$. Für XTR ist außerdem der Begriff einer Untergruppe wichtig. Eine Untergruppe besteht aus einer Teilmenge der Obergruppe. Sie muss jedoch in sich bezüglich der gewählten Verknüpfung wieder eine eigene Gruppe bilden.

Für das DH-Schlüsselaustauschverfahren einigen sich die Kommunikationspartner \emph{A} und \emph{B} nun über einen unsicheren abhörbaren Kommunikationskanal auf
einen Generator \emph{g} und eine Primzahl \emph{p} und können so die zuvor eingeführte zyklisch multiplikative Gruppe $Z_{p}^*$ bilden.
Teilnehmer \emph{A} berechnet nun $X = g^a \bmod p$ mit einem zufälligen Wert für \emph{a} und schickt \emph{X} über den unsicheren Kanal zu \emph{B}.
Teilnehmer \emph{B} berechnet $Y = g^b \bmod p$ und schickt \emph{Y} zu \emph{A}. Teilnehmer \emph{A} und \emph{B} können sich jetzt den geheimen Schlüssel $Y^a = g^{ab} \bmod p$ bzw.
$X^b = g^{ab} \bmod p$ berechnen, der zukünftig für eine symmetrische Verschlüsselung verwendet werden kann.

Ein möglicher Angreifer konnte während des Vorgangs also folgendes über den unsicheren Kanal mithören:
 \emph{g}, \emph{p}, \emph{$X = g^a \bmod p$}, \emph{$Y = g^b \bmod p$}. 
Um den geheimen Schlüssel $g^{ab} \bmod p$ zu rekonstruieren, müsste der Angreifer also entweder \emph{a} aus dem Wert $g^a \bmod p$ oder
 \emph{b} aus dem Wert $g^b \bmod p$ berechnen.
 
 Diese Ermittlung des Wertes ist als das \emph{diskrete Logarithmus Problem} bekannt. Die Sicherheit des DH-Schlüsselaustausches beruht also auf der Schwierigkeit
 $g^x \bmod p$ nach \emph{x} aufzulösen. Für die Bestimmung von \emph{x} ist hier noch kein effizientes Verfahren bekannt. Das Problem wird verschärft je größer die Primzahl \emph{p} gewählt wird.
 
 Allein für den Schlüsselaustausch werden also einige Daten über einen Kanal geschickt.
 Nach heutigem Stand gilt dieses Verfahren ab einer Mächtigkeit von 1024 Bit für $Z_p^*$  als sicher.
 
 Der DH-Schlüsselaustausch dient als Beispiel für eine mögliche Anwendung von XTR. Hier wird eine, für eine ausreichende Sicherheit, große zyklische multiplikative Gruppe verwendet um das diskrete Logarithmusproblem als unlösbar zu gestalten. XTR verallgemeinert das Konzept des zyklisch multiplikativen Gruppe mit der Einführung von Körpern und beschreibt ein Verfahren, mit dem ein großer Körper auf einen kleineren abgebildet werden kann.


\section{XTR zur Reduktion des Kommunikationsoverheads}
\label{sec:xtr}

\subsection{Grundlagen für XTR - Galois-Körper GF}
Das XTR Verfahren, "`Efficient and Compact Subgroup Trace Representation"' beschäftigt sich mit der Reduktion des Kommunikationsoverheads bei möglichst gleichbleibender Sicherheit, also der Minimierung der zu übertragenden Bits bei angewandten Kryptografieverfahren wie z.B. dem kurz vorgestelltem DH-Schlüsselaustausches.
Dieses Verfahren arbeitet mit endlichen Körpern, sogenannten Galois Körpern \emph{GF}. Ein Körper ist strenger definiert als die zuvor eingeführte Gruppe. Ein Körper \emph{K} besitzt die Addition "`+"' und die Multiplikation "`$\cdot$"' als Verknüpfungen mit den folgenden Eigenschaften:

\begin{itemize}
	\item $(K,+)$ ist eine abelsche Gruppe mit neutralem Element 0, also wie die zuvor definierte Gruppe mit zusätzlichem Kommutativgesetz
	\item $(K \backslash \{0\},\cdot)$ ist eine ablesche Gruppe mit neutralem Element 1
	\item Distributivgesetz für alle Elemente aus K
\end{itemize}

Ein endlicher Körper (Galois Körper, GF), ist ein Körper mit einer endlichen Anzahl von Elementen.
Für jede Primzahl $p$ und jede natürliche Zahl $n$ existiert ein Körper mit $p^n$ Elementen. Er wird mit $F_{p^n}$ oder $GF(p^n)$ bezeichnet. Hierbei ist die Addition und Multiplikation wie gewohnt  modulo p auszuführen (Ansonsten wäre der resultierende Körper nicht endlich, da eine Addition von zwei ausgewählten Elementen ein "`neues"' Element des Körpers ergeben kann).

Wie bereits erwähnt, hängt die Sicherheit der Anwendung eines Verfahrens mit diskretem Logarithmus-Problem maßgeblich von der Anzahl der Elemente in so einem Körper $GF(p^n)$, also von $p^n$ ab. Der XTR-Algorithmus ist in der Lage beliebige Exponenten des Generators $g$ des Körpers $GF(p^6)$ über $GF(p^2)$ darstellen zu können. Außerdem kann die Exponentiation in diesem Teilkörper effizient berechnet werden.

\subsection{Funktionsweise des XTR-Verfahrens}

Die Artefakte von XTR sind wie erwähnt  $GF(p^6)$ und $GF(p^2)$. Es wird jetzt ein Generator $g \in GF(p^6) $ mit der Ordnung $ q > 6$ gewählt. Die Ordnung von $g$ ist definiert als die kleinste ganze Zahl $n$ für die gilt: $g^n \bmod p = 1$. Außerdem muss $g$ den Wert $p^2 - p +1$ teilen. Das so gewählte $g$ erzeugt nun eine Untergruppe von $GF(p^6)$ mit der Ordnung $q$. Für XTR muss im Gegensatz zu vielen anderen Verfahren weder diese Untergruppe $<g>$, noch Elemente aus $GF(p^6)$ erzeugt werden. Stattdessen reicht eine Repräsentation von $GF(p^2)$ aus.

Die auszuwählende Primzahl $p$ hat die Eigenschaft $p \equiv 2 \bmod 3$, $p \equiv 2 \bmod 3$ generiert also $GF(3)*$. Daraus folgt, dass sich Elemente aus $GF(p^2)$ mit $x_1\alpha + x_2\alpha^p = x_1\alpha + x_2\alpha^2 $ darstellen lassen, wobei $x_1,x_2 \in GF(p)$ und $\alpha, \alpha^p$ die Nullstellen des nicht reduzierbaren Polynoms $X^2 + X + 1$ sind.

Für Kryptosysteme müssen je nach Verfahren die verschiedensten Berechnungen durchgeführt werden. Für Elemente aus $GF(p^2)$ ergeben sich nun folgende Komplexitäten für $x,y \in GF(p^2)$:
\begin{itemize}
	\item $x^p = x_1^p\alpha^p + x_2^p\alpha^{2p} = x_1\alpha + x_2\alpha^2$. Die Berechnung ist also kostenlos.
	\item $x^2$ kann auf zwei Quadrierungen und eine Multiplikation in $GF(p)$ zurückgeführt werden
	\item $x \cdot y$ benötigt drei Multiplikationen in $GF(p)$
\end{itemize}
Diese Analyse kann man ebenfalls für die Elemente $x,y \in GF(p^6)$ mit gleichbleibendem $p$ durchführen. Hier zeigt sich ein deutlicher Berechnungskomplexitätsunterschied.
So benötigt die Quadrierung in $GF(p^6)$ ca. 14 Multiplikationen in $GF(p)$, die Multiplikation $x \cdot y$ sogar 18.

Die Berechnungen in $GF(p^2)$ sind also wesentlich effizienter als in $GF(p^6)$.

$GF(p^2)$ (im weiteren als $K$ bezeichnet) ist offensichtlich ein Unterkörper von $GF(p^6)$ (im weiteren Verlauf als $L$ bezeichnet), da $K\subseteq L$.
Das Paar $L$ und $K$ wird als Körpererweiterung bezeichnet ($L\backslash K$).

Um die Elemente aus L ($GF(p^6)$) mit Elementen aus K ($GF(p^2)$) darzustellen, verwendet XTR die Spur(engl. Trace, Abkürzung: Tr) von L nach K. Mathematisch ist die Spur einer Körpererweiterung eine lineare Abblidung von L nach K, also die Zuordnung von Elementen des größeren Körpers L auf den kleineren Körper K. 

In \cite{xtr-explanation} wird gezeigt, dass alle Elemente von $g^n \in GF(p^6)$ mit der Hilfe der Spur $Tr(g^n)$ dargestellt werden können, wobei diese Spur in $GF(p^2)$ liegt. Es wird also nur noch mit den Spuren gerechnet. Um diese Darstellung in kryptographischen Verfahren verwenden zu können, muss $Tr(g^n)$ über ein bekanntes $Tr(g)$ bestimmt werden können (Entspricht der Exponentiation von einem bekannten $g$ zu $g^n$). Es existiert ein Algorithmus, der folgende Berechnung für beliebige $n$ ermöglicht:\newline
$S_n(Tr(g)) = (Tr(g^{n-1}), Tr(g^n),  Tr(g^{n+1}))$\newline
Hiermit lässt sich aus $Tr(g)$ u.a. ein beliebiges $Tr(g^n)$ bestimmen. Eine entsprechende Analyse zeigt außerdem, dass die Berechnung von $Tr(g^n)$ drei mal so schnell ist wie die Berechnung von $g^n$. Das Ergebnis eines beliebigen $Tr(g^n)$ befindet sich nachweislich im Körper $GF(p^2)$, die Elemente der Berechnung $g^n$ aber in $GF(p^6)$. Das Sicherheitsniveau aufgrund der Mächtigkeit von $GF(p^6)$ bleibt also mit dieser Methode erhalten.

Das Ziel, eine größere Gruppe $g^n$ mit einer kleineren Gruppe darzustellen wird also durch die Verwendung der Spur erreicht. 
In einigen kryptografischen Verfahren wird allerdings nicht nur eine Berechnung von $g^n$ benötigt sondern meistens auch ein Produkt von zwei oder mehreren Exponentiationen von $g$, also beispielsweise $g^a \cdot g^b$. Bei der Verwendung der klassichen Repräsentation ist diese Rechnung trivial und einfach durchzuführen ($g^a \cdot g^b = g^{a+b})$. Die Berechnung über die Spur von $g$ ist dabei schwieriger. Die Publikation \cite{xtr-explanation} zeigt jedoch ein Verfahren mit dem $Tr(g^a \cdot g^b)$ effizient berechnet werden kann, sogar mit einem Effizienzzuwachs von 75\% gegenüber der Berechnung von $g^a \cdot g^b$. 

\subsection{Wahl der benötigten Parameter für eine praktische Realisierung von XTR}
\label{subsec:paramwahl}
Eine wichtige Frage bei der praktischen Umsetzung des XTR-Verfahrens betrifft die Wahl von $p$, der Untergruppengröße $q$ von $GF(p^6)$ sowie einem geeigneten $Tr(g)$. Das Resultat wird als "`XTR group"' bezeichnet.
Nach \cite{xtr-explanation} muss $q$ das Polynom $p^2-p+1$ teilen.
Verglichen wird hier die Wahl der Parameter mit einer zu RSA vergleichbaren Sicherheit von 1024 Bit. Aus $p$ wird der Körper $GF(p^6)$ erzeugt.
Dieser soll dementsprechend eine Mächtigkeit von ebenfalls 1024 Bit aufweisen. Somit ergibt sich für $p$ eine ungefähre Länge von $1024/6\approx170$ Bit. Aufgrund aktueller kryptoanalytischen Methoden sollte $p$ nicht viel kleiner als $q$ gewählt werden, also beide ca. 170Bit.
Die Parameter können über folgende Gleichungen ermittelt werden:
\begin{itemize}
\item Finde ein $r$ mit den Eigenschaften: $q$ ist eine Primzahl der Länge 170 Bit und $q=r^2-r+1$  
\item Danach muss ein $k\neq 1$ ermittelt werden für das gilt: $p=r+k\cdot q$, wobei $p$ wieder eine Länge von ca. 170 Bit besitzen sollte und außerdem $p \equiv 2 \bmod 3$ gilt. Bei der Wahl von $k \equiv 1$ vereinfacht sich die Lösung des diskreten Logarithmus-Problems und ist somit zu vermeiden.
\item Ermittlung eines $Tr(g)$ wie in \cite{xtr-explanation} beschrieben.
\end{itemize}

Das "`public key"' Element des XTR Verfahrens ist somit das Tripel $(p,q,Tr(g))$ und kann zwischen den Kommunikationspartnern ausgetauscht werden.
Wie beschrieben besteht es aus dem finiten Feld (generiert aus $p$), der Ordnung der Untergruppe ($q$) sowie dem Generator $Tr(g)$.
Das XTR Verfahren unterstützt auch Protokolle, die private Schlüssel vorsehen, sowie die digitale Signierung. 
Ein privater Schlüssel hat dann die Form $Tr(g^k)$, wobei $k$ hier der geheime Teil ist (denn $Tr(g)$ ist öffentlich). 

Die explizite Anwendung des XTR Verfahrens auf speizelle Kryptosysteme wie den DH-Schlüsselaustausch, die ElGamal Verschlüsselung sowie die Nyberg-Rueppel Signatur lässt sich in \cite{xtr-explanation} nachlesen.

\section{Sicherheit von XTR}
\label{sec:sicherheit}
\subsection{Diskreter Logarithmus der XTR-Artefakte}
Das in \ref{sec:dh} erwähnte Logarithmus Problem muss hinsichtlich XTR auf den diskreten Logarithmus in $GF(p^t)$ angepasst untersucht werden.
Dabei sei $<\gamma>$ eine Untergruppe der multiplikativen Gruppe $GF(p^t)$* der Ordnung $\omega$. Es lassen sich verschiedene Probleme bezüglich des Diffie-Hellman Schlüsselaustausches definieren:

\begin{itemize}
\item Diffie-Hellman Problem (DH, auch CDH genannt): Die Berechnung von $\gamma^{xy}$ aus $\gamma^x$ und $\gamma^y$, geschrieben als $DH(\gamma^x,\gamma^y) = \gamma^{xy}$
\item Diffie-Hellman Decision (DHD, auch DDH genannt): Die Entscheidung ob für $a,b,c \in <\gamma> , c=DH(a,b)$ gilt, also die Zuordnung von $c$ zu gegebenen $a$ und $b$
\item Diskretes Logarithmus Problem (DL): Die Berechnung von $x$ bei einem gegebenen $a = \gamma^x$, geschrieben als $x=DL(a)$
\end{itemize}
Es ist hier die allgemeine Vermutung dass die Unlösbarkeit des DL-Problems die Unlösbarkeit von DH und DHD impliziert.
Eine Analyse der möglichen Angriffsmethoden zeigt die Abhängigkeit des DL-Problems von der Mächtigkeit von $<\gamma>$ sowie der Größe von $\omega$. Aufgrund der Verwendung der in \ref{subsec:paramwahl} eingeführten XTR group ist das Problem nur für diese Untergruppe zu lösen.
In \cite{xtr-explanation} ist jedoch gezeigt dass bei einer geeigneten Wahl von $p$ und $q$ von ca. 170 Bits das DL Problem in der XTR group schwieriger zu lösen ist als die Faktorisierung in RSA mit 1020 Bits.

In XTR wird jedoch mit den Spuren gerechnet so dass die Probleme DL, DH und DHD wie folgt speziell auf XTR umgeschrieben werden können.

\begin{itemize}
\item XTR-DH Problem: Berechnung von $Tr(g^{xy})$ aus $Tr(g^x)$ und $Tr(g^y)$, geschrieben $XDH(g^x, g^y) = g^{xy}$
\item XTR-DHD: Entscheidung ob $a,b,c \in Tr(<g>) , c=XDH(a,b)$ gilt
\item XTR-DL: Bestimmung von $x$ aus $a = Tr(g^x)$, geschrieben als $x = XDL(a)$
\end{itemize}

Bei einer genaueren Untersuchung zeigt sich hier, dass die Algorithmen zur Lösung von DL, DH oder DHD auch zur Lösung von XDL, XDH und XDHD
mit überschaubarem Aufwand umgeschrieben werden können.
In der Praxis genügt also ein XTR-DH Algorithmus um ein DL Problem zu lösen. Bei der Wahl der vorgeschlagenen Parameter aus \ref{subsec:paramwahl}
ist die Lösung aber nachweislich mit der Sicherheit von RSA zu vergleichen.

\subsection{Seitenkanalangriffe auf XTR}
\label{subsec:seitenkanal}
Seitenkanalangriffe versuchen über die Beobachtung der physikalisch auftretenden Effekte eines rechnenden Prozessors, Rückschlüsse auf die ausgeführten Anweisungen zu erhalten. Oft anzutreffen sind die Seitenkanalangriffe, die den Energieverbrauch des Prozessors aufzeichnen und auf diese Art ausgeführte Befehle zu extrahieren. Zwei bekannte Verfahren sind hier"`Simple Power Analysis (SPA)"' und "`Differential Power Analysis (DPA)"'. SPA nutzt nur Informationen aus einem einzelnen beobachteten Effekt, DPA verwendet mehrere Messquellen. Da bei der Verwendung von Smart-cards die Energieversorgung über ein externes Lesegerät drahtlos erfolgt, lässt sich hier beispielsweise die Energieverbrauchskurve leicht ermitteln.
Ein Angreifer kann nun analysieren, welche Energiekurven zu welcher Operation gehören und beim Ablauf des Algorithmus eine Operations-Spur erzeugen, also einen rekonstruierten Kontrollfluss.

Für die Seitenkanalangriffsanalyse bezüglich XTR, werden die benötigten Operationen untersucht. 
Da XTR auf der Berechnung der Spur $Tr(g^n)\in GF(p^2)$ basiert, wird die Abkürzung $c_n = Tr(g^n)$ eingeführt, wobei $c_0=3$. Die Berechnung von $Tr(g^n)$ wird "`Single Exponentiation (XTR-SE)"' genannt.
Es lassen sich als Konsequenz folgende Rechenregeln aufstellen:
\begin{itemize}
	\item $c_{-n}=c_{np}=c_n^p$
	\item $c_{2n}=c_n^2-2c_n^p$, im weiteren als \newline $XTRDBL(c_n)$ bezeichnet
	\item $c_{n+2}=c_1 \cdot c_{n+1} - c_1^p \cdot c_n + c_{n-1}$, im weiteren als\newline  $XTR\_C_{n+2}(c_{n-1},c_n,c_{n+1},c_1)$ bezeichnet
	\item $c_{2n-1}=c_{n-1}\cdot c_n - c_1^p \cdot c_n^p + c_{n+1}^p$, im weiteren als\newline  $XTR\_C_{2n-1}(c_{n-1},c_n,c_{n+1},c_1)$ bezeichnet
	\item $c_{2n+1}=c_n \cdot c_{n+1} - c_1 \cdot c_n^p + c_{n-1}^p$, im weiteren als\newline  $XTR\_C_{2n+1}(c_{n-1},c_n,c_{n+1},c_1)$ bezeichnet
\end{itemize}
Zuletzt lassen sich hier noch die benötigten Multiplikationen der einzelnen Operationen in $GF(p)$ ermitteln. $XTRDBL(c_n)$ benötigt eine Multiplikation,
$XTR\_C_{n+2}$, $XTR\_C_{2n-1}$ sowie $XTR\_C_{2n+1}$ jeweils drei Multiplikationen.\cite{side-channel2}


\subsubsection{SPA Seitenkanalangriff}
Für eine Berechnung von XTR-SE werden die Operationen ($XTRDBL(c_n)$, $XTRDBL(c_n)$, $XTR\_C_{2n-1}$) oder ($XTRDBL(c_n)$, $XTRDBL(c_n)$, $XTR\_C_{2n+1}$) je nach zu berechnendem Wert wiederholt ausgeführt. Da aber $XTR\_C_{2n-1}$ sowie $XTR\_C_{2n+1}$ die gleiche Anzahl an Multiplikationen in $GF(p)$ benötigen, sind die ausgeführten Instruktionen unabhängig von dem geheimen Schlüssel (dem Exponenten aus $Tr(g^x)$).
XTR-SE ist also unempfindlich und somit sicher gegenüber SPA Seitenkanalangriffen.

%\subsubsection{Data-Bit DPA Seitenkanalangriff (DDPA)}
%DDPA ist eine besondere Form des DPA und beschäftigt sich mit der Repräsentation der Bits während des Algorithmus zur Berechnung von XTR-SE.
%Sobald der Angreifer die Funktionsweise des Algorithmus kennt, kann er die konsumierte Energie mit den veränderten Bits korrelieren und sich so geschickt die Bits des zu Exponenten der Berechnung rekonstruieren.

\subsubsection{Address-Bit DPA Seitenkanalangriff (ADPA)}
ADPA nutzt die mögliche Korrelation eines geheimen Wertes und die verwendeten Register bei der Abarbeitung des Wertes. So befindet sich 
in der Berechnung von XTR-SE aus \cite{xtr-explanation} der folgende Teilausschnitt wobei $m_i$ das i-te Bit des geheimen Wertes darstellt. $T[x]$ ist ein Feld für die Zwischenspeicherung von Werten und $C[x]$ Adressen.

\begin{itemize}
\item Falls $m_i$ = 0: Aktualisiere T[1] mit dem Register C[0] und T[2] mit dem Register C[1]
\item Falls $m_i$ = 0: Aktualisiere T[1] mit dem Register C[1] und T[2] mit dem Register C[2]
\end{itemize}

Verknüpft der Angreifer nun die DPA Analyse mit den entsprechenden Registern C[0] und C[1], so kann er sich den geheimen Wert $m$ ermitteln.
XTR-SE ist also hinsichtlich dieses Angriffes nicht sicher.\cite{side-channel2}

\subsubsection{Verdoppelungsattacke}
Die Verdoppelungsattacke basiert auf dem Ansatz dass ein Angreifer zwar nicht den Wert X einer Operation (z.B $2\cdot X$) ermitteln kann, jedoch kann er prüfen, ob zwei Operationen das gleiche Ergebnis liefern (Beobachtetes Ergebnis X = Beobachteten Ergebnis Y).
$S_n$ steht für die Berechnung des Tripels $(c_{n-1},c_{n},c_{n+2})$ bei gegebenem $c_1$ (Zur Erinnerung: $c_n = Tr(g^n)$).
Jetzt werden die Operationen verglichen bei der Berechnung von einem $S_n$ bei gegebenem $c_1$ und einem $\tilde{S}_n$ wenn $\tilde{c}_1 = c_2$.
Der private Wert $m$ wird im XTR-SE Algorithmus mit einer For-Schleife mit Index $i$ abgearbeitet. Hierbei lässt sich beobachten dass die XTRDBL Operation im Schritt $i$ bei der Berechnung von $S_n$ im Falle eines 0-Bits von $m$ die gleiche Operation wie bei der Berechnung von $\tilde{S}_n$ im Schritt $i+1$ ist. Mit der Hilfe dieses Verfahrens lässt sich als Konsequenz mit zwei Anfragen der private Teil $m$ ermitteln.\cite{side-channel2}



\section{Fazit}
Wie in dieser Arbeit zusammengefasst, ist XTR ein geeignetes Verfahren um effiziente Verschlüsselung durchgängig verwenden zu können. 
Bei einer immer weiter steigenden Rechenleistung von Desktop-Rechnern sowie Mobiltelefonen sollte man meinen, die benötigte Rechenleistung eines
Verschlüsselungsverfahren spiele keine Rolle. In Hinblick auf Smard-Cards, NFC-Anwendungen und ähnliches ist die Effizienz sowie die erforderliche Bandbreite jedoch ein entscheidendes Kriterium. Der komplexe Sachverhalt des Hintergrundes von XTR kann für die praktische Anwendung aufgrund
einfacher Algorithmen gut abstrahiert werden so dass sich die Einstiegshürde bei der Verwendung des Verfahrens reduziert. Auch die Wahl geeigneter Schlüssel wird dem Implementierendem weitestgehend abgenommen sodass die Fehleranfälligkeit für eine falsche Implementierung des Kryptografieverfahrens minimiert wird. Der hohe Effizienzzuwachs der Rechenoperationen in der XTR Darstellung von bis zu 600\% beeindruckt vor allem mit dem Hintergrund der nachweislich äquivalenten Sicherheit gegenüber vergleichbaren Verfahren. Die Unempfindlichkeit von XTR gegenüber SPA Seitenkanalattacken sprechen außerdem für das Verfahren. Mit möglichen, hier nicht vorgestellten, Gegenmaßnahmen für einen ADPA sowie einen Verdoppelungsangriff, präsentiert sich XTR als ein solides, resourcenschonendes und vor allem sicheres Verschlüsselungsverfahren mit einem breiten Anwendungsgebiet.


\ifCLASSOPTIONcaptionsoff
  \newpage
\fi




\bibliographystyle{IEEEtran}

\begin{thebibliography}{1}

\bibitem{xtr-explanation}
 A.~Lenstra and E.~R. Verheul \emph{The XTR Public Key System}, \relax Mendham, N.A and Eindhofen, Netherlands 2000.
\bibitem{side-channel2}
D.~Hand, J.~Lin and K. Sakurai \emph{On Security of XTR Public Key Cryptosystems Against Side Channel Attacks}, \relax Korea, Seoul and Japan, Fukuoka 2004
\bibitem{gruppe}
H.W.~Lang \emph{Mathematische Grundlagen - Gruppe}, \relax \newline Available from: "`http://www.iti.fh-flensburg.de/lang/algorithmen/grundlagen/gruppe.htm"',
28.08.2000
\bibitem{mulgruppe}
H.W.~Lang \emph{Mathematische Grundlagen - multiplikative Gruppe modulo n}, \relax \newline Available from: "`http://www.iti.fh-flensburg.de/lang/krypto/grund/gruppezn.htm"',
28.08.2000



\end{thebibliography}

\end{document}


